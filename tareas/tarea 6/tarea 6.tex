\documentclass[11pt,a4paper]{article}
\usepackage[utf8]{inputenc}
\usepackage{amsmath}
\usepackage{amsfonts}
\usepackage{amssymb}
\author{Vargas Diaz Marco Antonio }
\title{diseño de un modulacion de ancho de pulso (PWM) con Amp-Op y transistores}
\begin{document}
\maketitle
\subsection{modulacion de ancho de pulso}
La modulacion por ancho de pulso o PWM se usa para controlar el ancho de una señal digital con el propósito de controlar a su vez la potencia que se entrega a ciertos dispositivos. Modificando el ancho del pulso activo (que está en On) se controla la cantidad de corriente que fluye hacia el dispositivo.

\subsubsection{un PWM funciona}
como un interruptor, que constantemente se activa y desactiva, regulando la cantidad de corriente y por ende de potencia, que se entrega al dispositivo que se desea controlar. Éstos dispositivos pueden ser motores CC o fuentes de luz en CC, entre otros.
Si un motor es alimentado con 12 voltios, recibe todo el tiempo la corriente que este pide y entrega la máxima potencia, si es alimentado con 0 voltios, no recibe corriente y no obtiene potencia.
En un sistema PWM el motor recibe corriente por un tiempo y deja de recibirlo por otro, repitiéndose este proceso continuamente. Si se aumenta el tiempo en que el pulso está en nivel alto (12 V en nuestro ejemplo), se entrega más potencia y si se reduce el tiempo entrega menos potencia.
 
\subsection{control de intensidad de iluminacion de unos leds}
Éste circuito utiliza el conocido temporizador 555. Éste circuito integrado es conectado como multivibrador astable, y entrega en su salida una onda cuadrada.
El ciclo de trabajo de la onda de salida se controla con el potenciómetro, el transistor se encarga de activar y desactivar la carga (en este caso dos LEDs), según el ciclo de trabajo.
Modificando el ciclo de trabajo se puede pasar de apagado total (ciclo del 0porciento) a encendido total (ciclo del 100porciento). Si se desea una intensidad de luz intermedia se puede utilizar un ciclo del 50porciento o un porcentaje cercano.
De la misma manera también se puede controlar la velocidad del motor CC.
\subsection{control de velocidad de un motor CC}
El siguiente circuito es igual al anterior, pero se han cambiado los LED y la resistencia de 100 ohmios  por un motor CC con un diodo en paralelo pero invertido, con el propósito de proteger el transistor.
La velocidad del motor CC cambiará con la variación del ciclo de trabajo escogido con el potenciómetro.
Para amplicaciones en corriente alterna, se puede optar por un dimmer o control de velocidad de motores en CA/AC
\end{document}
