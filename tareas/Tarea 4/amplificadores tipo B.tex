\documentclass[10pt,a4paper]{article}
\usepackage[utf8]{inputenc}
\usepackage{amsmath}
\usepackage{amsfonts}
\usepackage{amssymb}
\usepackage{graphicx}
\usepackage[left=2cm,right=2cm,top=2cm,bottom=2cm]{geometry}
\author\part{Marco Antonio Vargas Diaz}
\begin{document}
\section{amplificadores tipo B}
los amplificadores de clase B se caracterizan por tener intensidad casi nula a traves de sus transistores del circuito,por lo que en reposo el consumo es casi nulo.
\includegraphics[width=6cm]{images (1).jpeg} 
\section{caracteristicas}
se les denomina amplificadores clase B,cuando el voltaje de polarizacion y la maxima amplitud de la señal entrante poseen valores que se hacen que la corriente de salida circule durante el semiciclo de la señal de entrada.
la caracteristicas principales de este tipo de amplificadores es el alto factor de amplificacion.

Circuito Amplificador de Transformador Push-pull de Clase B


El circuito anterior muestra un circuito amplificador Clase B estandar que utiliza un transformador de entrada con toma central compensada, que divide la señal de forma de onda entrante en dos mitades iguales y que están desfasadas 180grados entre sí. Otro transformador con toma central en la salida se utiliza para recombinar las dos señales que proporcionan la mayor potencia a la carga. Los transistores utilizados para este tipo de circuito amplificador push-pull de transformador son ambos transistores NPN con sus terminales de emisor conectados entre sí.

Aquí, la corriente de carga se comparte entre los dos dispositivos de transistor de potencia a medida que disminuye en un dispositivo y aumenta en el otro a lo largo del ciclo de señal, reduciendo el voltaje y la corriente de salida a cero. El resultado es que ambas mitades de la forma de onda de salida ahora oscilan desde cero hasta el doble de la corriente de reposo, reduciendo así la disipación. Esto tiene el efecto de casi duplicar la eficiencia del amplificador a alrededor del 70%.

amplifiadore clase B estos basiamente son la mezcla de los dos anteriores.cuando el voltaje de polarizacion y la maxima amplitud de la señal entrante posee n valores que se hacen que la corriente de salida circule durante menos del ciclo completo y mas de la mitad del ciclo de la señal de entrada, se les denomina: amplificadores de potencia clase B
dado que ocupa un lugar intermedio entre los de clase AB, cuando el voltaje de la señal es moderado funciona como uno de clase A, cuando la señal es fuerte se desempeña como una de clase b, con una eficiencia y deformacion moderadas.
\includegraphics[width=6cm]{images.jpeg} 
\section{curvas del comportamiento del amplificador B}
El amplificador de clase B tiene la gran ventaja sobre sus primos de amplificador de clase A en que ninguna corriente fluye a través de los transistores cuando están en estado de reposo (es decir, sin señal de entrada), por lo tanto no se disipa potencia en los transistores de salida o transformador cuando no hay señal presente a diferencia de las etapas de amplificador de Clase A que requieren un sesgo de base significativo, disipando así gran cantidad de calor, incluso sin presencia de señal de entrada.

Por lo tanto, la eficiencia total de conversión del amplificador es mayor que la de la Clase A equivalente, alcanzando eficiencias tan altas como 70%, lo que resulta en casi todos los tipos modernos de amplificadores push-pull operados en este modo Clase B.

Amplificador de empuje y tracción de clase B sin transformador

Una de las principales desventajas del circuito amplificador Clase B anterior es que utiliza transformadores de derivación central equilibrada en su diseño, por lo que es costoso de construir. Sin embargo, hay otro tipo de amplificador de Clase B llamado Amplificador de Clase B de Simetría Complementaria que no usa transformadores en su diseño, por lo tanto, es sin transformador usando pares de transistores de potencia complementarios o coincidentes.

Como los transformadores no son necesarios, esto hace que el circuito del amplificador sea mucho más pequeño para la misma cantidad de salida, también no hay efectos magnéticos extraños o distorsión del transformador para afectar la calidad de la señal de salida. A continuación, se proporciona un ejemplo de un circuito amplificador Clase B "sin transformador".
\end{document}